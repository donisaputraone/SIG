Sejarah 
Antartika 
berawal
dari
teori
dunia 
barat 
tentang
benua
yang
membentang
luas, 
yang
dikenal
sebagai 
Terra Australis, 
yang 
diyakini berada di ujung selatan dunia. Istilah Antartika, mengacu pada kebalikan dari Lingkar Arktik diciptakan oleh Marinus dari Tirus pada abad ke-2 Masehi.
Melingkar Tanjung Harapan dan Tanjung Horn pada abad ke-15 dan ke-16 membuktikan bahwa Terra Australis Incognita ("Tanah Selatan yang Tidak Diketahui"), jika masih dapat dijumapai adalah sebuah benua dalam dirinya sendiri. Pada tahun 1773 James Cook dan krunya untuk pertama kalinya menyeberangi Lingkar Antartika meskipun mereka menemukan pulau-pulau terdekat, tetapi mereka tidak dapat melihat Antartika. Hal ini diyakini James Cook hanya berjarak sejauh 150 mil dari daratan Antartika.
Pada tahun 1820, beberapa penjelajah mengaku menjadi orang pertama yang telah melihat lapisan es yaitu Ekspedisi Rusia yang dipimpin oleh Fabian Gottlieb von Bellingshausen dan Mikhail Petrovich Lazarev, ekspedisi Inggris yang dipimpin oleh Edward Bransfield dan Nathaniel Brown Palmer. Lebih dari setahun kemudian John Davis mengklaim telah melakukan pendaratan pertama di benua selatan.
Ekspedisi pada awal abad ke-20 dikenal sebagai 'Abad Heroik Eksplorasi Antartika' untuk mencapai kutub selatan memiliki risiko cedera bahkan kematian. Seperti Roald Amundsen dari Norwegia mencapai Kutub pada tanggal 14 Desember 1911, namun Ia menghilang pada bulan Juni 1928 ketika berpartisipasi dalam sebuah misi penyelamatan di Laut Barents.

